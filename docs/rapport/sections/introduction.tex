\section{Introduction}

Ce document présente une étude expérimentale portant sur l’implémentation du jeu de la \emph{Bataille Navale}.  
L’objectif principal est d’analyser et de comparer plusieurs stratégies de tir (principalement heuristiques) afin d’évaluer leurs performances selon différents indicateurs : taux de victoire, efficacité moyenne (nombre de coups nécessaires), stabilité des résultats et comportement selon les configurations de jeu.

\medskip

L’ensemble du projet est organisé comme suit :

\begin{itemize}
  \item \textbf{Code source} : disponible dans le répertoire \texttt{src/}, comprenant les modules de simulation, les stratégies testées et les outils d’expérimentation.
  \item \textbf{Résultats expérimentaux} : les fichiers générés (CSV) sont automatiquement déposés dans \texttt{Results/}. Ils regroupent les données exploitées pour l’analyse : résumés de tournois, mesures de self-play, statistiques de performance, etc.
  \item \textbf{Documentation technique} : accessible dans \texttt{docs/}, elle inclut la description des classes, modules et structures internes du projet. Selon le format final du rapport (PDF ou HTML), ces pages peuvent être référencées ou intégrées sous forme de liens.
\end{itemize}

\medskip

L’étude qui suit repose sur une série d’expérimentations reproductibles, réalisées à partir des données contenues dans ce dépôt. Elle vise à mettre en évidence les différences de comportement entre stratégies, à dégager des tendances générales et à fournir une base comparative solide pour de futurs développements.
