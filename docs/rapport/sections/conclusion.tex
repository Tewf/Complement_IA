\section{Conclusion}

Cette étude a permis d'évaluer de manière systématique plusieurs stratégies appliquées au jeu de la Bataille Navale, en combinant analyses de tournoi et expérimentations en self-play. Les résultats se déclinent comme suit :

\paragraph{Observations principales}
\begin{itemize}
  \item Les approches fondées sur le \emph{placement counting} (stratégie Markov) offrent les meilleures performances globales : taux de victoire élevé, efficacité moyenne supérieure et excellente stabilité statistique.
  \item Les méthodes Monte Carlo présentent des performances proches mais requièrent un coût de calcul plus élevé et montrent une plus grande variabilité liée à l'échantillonnage.
  \item La stratégie hybride combine efficacement ciblage déterministe et exploration probabiliste, se montrant robuste et cohérente avec les tendances observées.
\end{itemize}

\paragraph{Pistes d'amélioration}
\begin{itemize}
  \item Optimiser le temps de calcul des approches Monte Carlo (algorithmes plus rapides ou parallélisation).
  \item Affiner la composante probabiliste via des modèles plus précis ou des heuristiques adaptées.
  \item Intégrer des stratégies apprenantes pour adapter la recherche à des configurations ou distributions de flotte variées.
  \item Étendre les expérimentations à d'autres tailles de grille ou schémas de placement afin d'obtenir une évaluation plus complète des comportements stratégiques.
\end{itemize}
