\section{Stratégies étudiées (heuristiques)}

Les stratégies implémentées dans le dépôt reposent sur deux composantes principales :
\begin{itemize}
  \item \textbf{Partie déterministe (Target)} : dédiée au \emph{ciblage} — exploitation des informations directes (cases touchées) pour déterminer l'orientation du navire et le couler.
  \item \textbf{Partie probabiliste (Hunt)} : destinée à l'exploration et à l'estimation des positions plausibles des navires (placement counting, Monte Carlo) pour orienter les tirs en phase de recherche.
\end{itemize}

Les principales heuristiques étudiées sont résumées ci-dessous.
% -------------------------------------------------------
\footnote{Les différentes heuristiques se trouvent dans ~: \texttt{src/heuristic/} et \texttt{src/joueurs/SmartBot.java}.}
%-------------------------------------------------------

% -------------------------------------------------------
\subsection{Uniforme (Uniform)}
% -------------------------------------------------------

Stratégie de base entièrement aléatoire.  
À chaque tour, elle choisit uniformément une case non encore visée.  
Elle sert de référence minimale (baseline) pour comparer les approches plus élaborées.

% -------------------------------------------------------
\subsection{Markov}
% -------------------------------------------------------

Stratégie probabiliste fondée sur le \emph{placement counting} :  
pour chaque case, l’algorithme compte le nombre de placements possibles de la flotte compatibles avec l’état du plateau (touchés, coulés, cases à l’eau).  
La case ayant la probabilité estimée la plus élevée d’appartenir à un navire est sélectionnée.

% -------------------------------------------------------
\subsection{Monte Carlo}
% -------------------------------------------------------

Méthode d’échantillonnage :  
un grand nombre de placements aléatoires de la flotte encore possible sont générés, en respectant toutes les contraintes observées.  
Une \emph{heatmap} est ensuite construite en comptabilisant la fréquence d’occupation de chaque case dans ces échantillons.  
La stratégie tire sur la case la plus fréquemment occupée.

% -------------------------------------------------------

\subsection{Smart / Hybride}
\label{subsec:smart}

Stratégie hybride combinant les deux composantes décrites ci‑dessus : la phase d'exploration \emph{Hunt} (probabiliste) et la phase de ciblage \emph{Target} (déterministe). Concrètement :

\begin{itemize}
  \item \textbf{Hunt (exploration probabiliste)} : la stratégie utilise des estimations (par placement counting ou Monte Carlo) pour repérer des zones prometteuses.
  \item \textbf{Target (ciblage déterministe)} : dès qu'une case est touchée, la stratégie applique un ciblage déterministe des cases adjacentes pour identifier l'orientation et couler le navire.
\end{itemize}
