\section{Définition du jeu et règles}

La \emph{Bataille Navale} est un jeu à deux joueurs se déroulant sur une grille carrée. Chaque joueur dispose d’une flotte composée de navires de longueurs prédéfinies et cherche à détruire la flotte adverse à l’aide de tirs successifs. Les hypothèses utilisées dans ce projet sont les suivantes :

\begin{itemize}
  \item La grille est de taille $N \times N$ (souvent $N = 10$).
  \item La flotte est composée de navires de longueurs fixes, par exemple \{5, 4, 3, 3, 2, 2\}.
  \item Les navires sont placés intégralement dans la grille, sans chevauchement, et orientés horizontalement ou verticalement.
  \item Les joueurs tirent à tour de rôle sur des cases encore non visées. Chaque tir est classé comme \textit{touché}, \textit{coulé} ou \textit{à l’eau}.
  \item La partie se termine lorsqu’un joueur a coulé tous les navires adverses.
\end{itemize}

Les variantes éventuelles (taille de grille différente, flotte réduite, règles alternatives) sont précisées dans la section méthodologique du rapport.